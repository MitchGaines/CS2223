++rticle}
\begin{document}

\hyphenation{Com-pu-ter-Pro-cess-able}

\begin{center}

{\small \bf Execution of the Tasks of the Metalanguage Subcommittee\\
of the Text Encoding Initiative}

{\small \bf Executive Summary}

S.A. Mamrak\\

\today

\end{center}

\section{Overview of the Text Encoding Initiative Project}

The Association for Computers and the Humanities, the Association for
Computational Linguistics, and the Association for Literary and
Linguistic Computing have proposed to develop and promote guidelines
for the preparation and interchange of machine-readable texts for
scholarly research.  Committees on text documentation, text
representation, text analysis and formal-language issues have been
formed to draft the guidelines.  A variety of other professional
organizations and learned societies are participating in the project's
advisory board and will approve the final form of the guidelines,
following which the three sponsoring organizations will arrange for
their publication.

Professor S. A. Mamrak has been invited to chair the committee
addressing formal-language issues, now designated the Metalanguage
committee.  This committee has two tasks.  The first is to generate
guidelines for the use of SGML Document Type Definitions in the TEI
project.  The second is to design an intermediate language, for
representing currently existing encoding schemes, to facilitate
translation into their SGML equivalents.

\section{Proposed Work of the Metalanguage Committee}

We are proposing to undertake the two tasks assigned to the
Metalanguage Committee, and in addition to begin the design and
implementation of a set of prototype software tools for creating and
manipulating DTDs.  These tools will be unique in that they will
enforce usage according to the guidelines promulgated by the
Metalanguage committee.  These guidelines will be generally
applicable, not only to the work of the TEI, but to the work of all
groups faced with specifying DTDs.  Thus, the toolset is expected to
be generally applicable as well.

In particular, we propose to begin this toolset with the design and
implementation of a software environment to support the specification
of the DTDs themselves.  This specification task is a very complex one,
and virtually impossible for humans to perform properly without software
tools that can aid in avoiding the specification of incomplete or
incorrect DTDs.  Existing tools to support this specification are not
nearly adequate to address all the forms of complexity.  We have
already fully identified all the sources of complexity in the task,
reviewed related work and found it lacking, and conceived a design for
an adequate support system.  We now propose to proceed with the
implementation of the prototype environment.

\end{document}

