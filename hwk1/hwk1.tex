
% homework template % 
%
% NOTE: 
% Be sure to define your name with the \name command 
% sure to use the \answer command for each of your answers 
%   (first argument: problem name 
%   second argument: collaborators (write 'none' if you solved it alone)) 
\documentclass[12pt]{article}

%\newcommand{\name}{ INSERT NAME HERE} 
%\newcommand{\pcroblemset}{ Homework X }
%\pagestyle{headings} 
\usepackage[dvips]{graphics,color} 
\usepackage{amsfonts} 
\usepackage{amssymb} 
\usepackage{amsmath} 
\usepackage{latexsym} 
\usepackage{enumerate} 
\setlength{\parskip}{1pc} 
\setlength{\parindent}{0pt} 
\setlength{\topmargin}{-3pc} 
\setlength{\textheight}{9.5in} 
\setlength{\oddsidemargin}{0pc} 
\setlength{\evensidemargin}{0pc} 
\setlength{\textwidth}{6.5in}

\newcommand{\answer}[2]{ \newpage \noindent \framebox{
	\vbox{
		Homework \hfill {\bf \problemset}
% 		\hfill \# #1 \\
			\name \hfill \today \\ 
%        Collaborators: #2
	}
} 
\bigskip
}


\begin{document}
Homework Assignment: 1\\ 
Name: Jonathan Gaines\\ 
Date: 3/22/2017\\ --------------------------------------------------------------------------------------------------
\begin {enumerate}
\item Summation Practice
				\begin {enumerate}[(a)]
\item $$\sum_{k=3}^{n+1} 1$   %a
					\item %b
					\item %c
					\item %d
					\item %e
					\item %f
					\item %g
					\item %h
				\end {enumerate}
\item Order of Growth 
\end {enumerate}

\end{document}
