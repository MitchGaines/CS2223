
% homework template % 
%
% NOTE: 
% Be sure to define your name with the \name command 
% sure to use the \answer command for each of your answers 
%   (first argument: problem name 
%   second argument: collaborators (write 'none' if you solved it alone)) 
\documentclass[12pt]{article}

%\newcommand{\name}{ INSERT NAME HERE} 
%\newcommand{\pcroblemset}{ Homework X }
%\pagestyle{headings} 
\usepackage[dvips]{graphics,color} 
\usepackage{amsfonts} 
\usepackage{amssymb} 
\usepackage{amsmath} 
\usepackage{latexsym} 
\usepackage{enumerate} 
\setlength{\parskip}{1pc} 
\setlength{\parindent}{0pt} 
\setlength{\topmargin}{-3pc} 
\setlength{\textheight}{9.5in} 
\setlength{\oddsidemargin}{0pc} 
\setlength{\evensidemargin}{0pc} 
\setlength{\textwidth}{6.5in}

\newcommand{\answer}[2]{ \newpage \noindent \framebox{
	\vbox{
		Homework \hfill {\bf \problemset}
% 		\hfill \# #1 \\
			\name \hfill \today \\ 
%        Collaborators: #2
	}
} 
\bigskip
}


\begin{document}
Homework Assignment: 1\\ 
Name: Jonathan Gaines\\ 
Date: 3/22/2017\\ --------------------------------------------------------------------------------------------------
\begin {enumerate}
\item Summation Practice
				\begin {enumerate}[(a)]
					\item $$\sum_{k=3}^{n+1} 1 = n-1 $$
					\item $$\sum_{i=1}^{100} (4+3i)  $$
								$$n(a_1 + \frac{d(n-1)}{2})   
									\left\{
										\begin{array}{2}
											a_1 = 7\\
											n = 100\\
											d = 3
										\end{array}
									\right
									\implies 100(7 + \frac{3(100-1)}{2}) = 15550 $$

					\item $$\sum_{i=1}^{200} (i-3)^{2} = \sum_{i=1}^{200}(i^{2}-6i+9)$$
								$$\sum_{i=1}^{200} i^{2} - 6(\sum_{i=1}^{200} i) + \sum_{i=1}^{200} 9 $$
							  $$= \frac{200(200+1)(400+1)}{6} - 6 \left\{ \frac{200(200+1)}{2} \right\} + 9(200) $$
								$$= 2567900 $$

					\item $$\sum_{i=10}^{80} (i^{3} + i^{2}) = \sum_{i=10}^{80} i^{3} + \sum_{i=10}^{80} i^{2} $$
								$$(\sum_{i=1}^{80} i^{3} - \sum_{i=1}^{9} i^{3})+(\sum_{i=1}^{80} i^{2} - \sum_{i=1}^{9} i^{2})$$
								$$= (\frac{80^{2}(80+1)^{2}}{4} - (\frac{9^{2}(9+1)^{2}}{4}) 
								+ (\frac{80(80+1)(160+1)}{6} - \frac{9(9+1)(18+1)}{6}) $$
								$$= 10669170 $$

					\item $$\sum_{j=0}^{n-1} j(j+1) = \sum_{j=0}^{n-1} j^{2} 
									+ 2(\sum_{j=0}^{n-1} j) 
									+ \sum_{j=0}^{n-1} 1 $$
									$$= \frac{n(n-1)(2n-1)}{6} + n(n-1) + (n-1) $$

					\item Create a summation for the following sequence: 2+4+8+16+32+64
								$$\sum_{i=1}^{6} 2^{i}$$	
					\item Create a summation for the following sequence: 2+6+18+54+162 
								$$\sum_{i=1}^{5} (3^{i} - 3^{(i-1)})$$		
				\item Create a summation for the following sequence: (-4)+(-1)+2+5+8+11+14
								$$\sum_{j=1}^{7} (3i - 7) $$	
				\end {enumerate}
\item Order of Growth
				\begin {enumerate}[(a)]
					\item $$\sum_{i=2}^{n-1} lgi^{2}$$ 
									$$= 2(\sum_{i=2}^{n-1} lg(i)) \implies \Theta(2(\sum_{i=2}^{n-1} lg(i))) = \Theta(lgi) $$
					\item $$\sum_{i=0}^{n-1} \sum_{j=0}^{i-1} (i+j) $$
									$$= \sum_{i=0}^{n-1} \sum_{j=0}^{i} (2i + j - 1) = \frac{1}{2}(\sum_{i=0}^{n-1} n^{2} + ...) \implies \Theta(n^3)$$
				\end {enumerate}
\item Time Efficiency Analysis
				\begin {enumerate}[(a)]
					\item This algorithm finds if neighboring values in the array are unique. 
					\item $$\sum_{i=0}^{n-2} \sum_{j=i+1}^{n-1} 1 = \sum_{i=0}^{n-2} (-i + n -1)$$
									$$= \frac{n(n-1)}{2} $$
					\item polynomial: $O(n^{2})$

				\end {enumerate}
\end {enumerate}

\end{document}
