
% homework template % 
%
% NOTE: 
% Be sure to define your name with the \name command 
% sure to use the \answer command for each of your answers 
%   (first argument: problem name 
%   second argument: collaborators (write 'none' if you solved it alone)) 
\documentclass[12pt]{article}

%\newcommand{\name}{ INSERT NAME HERE} 
%\newcommand{\pcroblemset}{ Homework X }
%\pagestyle{headings} 
\usepackage[dvips]{graphics,color} 
\usepackage{amsfonts} 
\usepackage{amssymb} 
\usepackage{amsmath} 
\usepackage{latexsym} 
\usepackage{enumerate} 
\setlength{\parskip}{1pc} 
\setlength{\parindent}{0pt} 
\setlength{\topmargin}{-3pc} 
\setlength{\textheight}{9.5in} 
\setlength{\oddsidemargin}{0pc} 
\setlength{\evensidemargin}{0pc} 
\setlength{\textwidth}{6.5in}

\newcommand{\answer}[2]{ \newpage \noindent \framebox{
	\vbox{
		Homework \hfill {\bf \problemset}
% 		\hfill \# #1 \\
			\name \hfill \today \\ 
%        Collaborators: #2
	}
} 

\bigskip
}

\newcommand\tab[1][1cm]{\hspace*{#1}}

\begin{document}
Homework Assignment: 2\\ 
Name: Jonathan Gaines\\ 
Date: 3/29/2017\\ --------------------------------------------------------------------------------------------------
\begin {enumerate}
\item XSort Algorithm
	\begin {enumerate}[(a)]
		\item
		$$EXAMPLE\Rightarrow 
			AXEMPLE\Rightarrow	
			AEXMPLE\Rightarrow
			AEEMPLX$$ $$\Rightarrow 
		  AEELPMX\Rightarrow
			AEELMPX $$		
		\item
			Time Efficiency: $O(n^{2})$ \\
			Space Efficiency: 
		\item 
			Stability of an algorithm refers to the handling of elements of the same value. A stable algorithm will swap the elements of the same value, but this is not always necessary. An example of where this is not necessary is this sort. It would however matter if a list of last names were being sorted, with the first name being used as a secondary sorting factor. This algorithm is not stable because the only instance where two elements swap places is when A < B. \par  
	\end {enumerate}
\item Bubble Sort
	\begin {enumerate}[(a)]
		\item
		$$EXAMPLE\Rightarrow 
			EAMPLEX\Rightarrow	
			AELEMPX\Rightarrow
			AEELMPX$$ $$\Rightarrow
			AEELMPX\Rightarrow
			AEELMPX $$
		\item
		\item
	\end {enumerate}
\item Show that $n^{2}$ \in $O(n^{2}+10n), n \geq 0$

\item Show that $n \not\in \Omega(n^{2})$ \\ \\
	\tab Choose $k=1$ \\ \\
	\tab Assuming $n > 1$, then \\ \\
	\tab $\frac{f(n)}{g(n)} = \frac{n}{n^{2}} < \frac{n^{2}}{n^{2}} = 1$ \\ \\
	\tab Choose $c=1$. Note that $n < n^{2}$ \\ \\
	\tab Thus $n \not\in \Omega(n^{2})$ because $n<n^{2}$ when $n>1$
\end {enumerate}

\end{document}
