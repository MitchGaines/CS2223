
% homework template % 
%
% NOTE: 
% Be sure to define your name with the \name command 
% sure to use the \answer command for each of your answers 
%   (first argument: problem name 
%   second argument: collaborators (write 'none' if you solved it alone)) 
\documentclass[12pt]{article}

%\newcommand{\name}{ INSERT NAME HERE} 
%\newcommand{\pcroblemset}{ Homework X }
%\pagestyle{headings} 
\usepackage[dvips]{graphics,color} 
\usepackage{amsfonts} 
\usepackage{amssymb} 
\usepackage{amsmath} 
\usepackage{latexsym} 
\usepackage{enumerate} 
\setlength{\parskip}{1pc} 
\setlength{\parindent}{0pt} 
\setlength{\topmargin}{-3pc} 
\setlength{\textheight}{9.5in} 
\setlength{\oddsidemargin}{0pc} 
\setlength{\evensidemargin}{0pc} 
\setlength{\textwidth}{6.5in}

\newcommand{\answer}[2]{ \newpage \noindent \framebox{
	\vbox{
		Homework \hfill {\bf \problemset}
% 		\hfill \# #1 \\
			\name \hfill \today \\ 
%        Collaborators: #2
	}
} 

\lstset{language=Python,
}

\bigskip
}


\begin{document}
Homework Assignment: 3\\ 
Name: Jonathan Gaines\\ 
Date: 4/19/2017\\ --------------------------------------------------------------------------------------------------
\begin {enumerate}
\item Job Optimization
	\begin {enumerate}[(a)]
		\item
			\begin{tabular}{ c | c | c | c }
					 Solution & Time Slot 1 & Time Slot 2 & profit \\
									1 & Job 1 			& Job 3				& 55 \\
									2 & Job 3				& Job 1				& 55 \\
									3 & Job 2 			& Job 1				& 65 \\
									4 & Job 2 			& Job 3				& 60 \\
									5 & Job 4				& Job 1				& 70 \\
									6 & Job 4				& Job 3				& 65 \\ 
									7 & Job 1				& N/A					& 30 \\
									8 & Job 2				& N/A					& 35 \\
									9 & Job 3 			& N/A					& 25 \\
									10& Job 4				& N/A					& 40
			\end{tabular}
		\item
			The optimal schedule has Job 4 in timeslot 1 and Job 1 in timeslot 2 for a profit of \$70. \par
		\item
			A high level greedy algorithm would choose the largest profit with a deadline of 1 or 2, then choose the largest profit with a deadline of 1. In this case, it would choose Job 4, then Job 1. \par
	\end {enumerate}
\item Dynamic Programming: Change Making
	\begin {enumerate}[(a)]
		\item
			The minimum number of coins needed to meet the amount is 3. \par
		\item
			Minimum coin combinations include \{1, 2, 5\} and \{3, 3, 3\}
		\item
			\begin{tabular}{c | c | c | c | c | c | c | c | c | c | c |}
										n & 0 & 1 & 2 & 3 & 4 & 5 & 6 & 7 & 8 & 9 \\
							$f(n)$	& 0	& 1 & 2 & 1 & 2 & 1 & 2 & 3 & 2 & 3
			\end{tabular}
		\item
			Change-making(D[j], n):
				f[0] = 0
				for i = 1 to n do
					temp = -1
					j = 1
					while j <= m and i >= D[j] do
						temp = min(f(i-D[j]), temp)
						j = j + 1
					f[1] = temp + 1
				return f(n)
	\end {enumerate}
\item Dyanmic Programming: Knapsack Problem
	\begin {enumerate}[(a)]
		\item
		\item
		\item
	\end {enumerate}
\item Greedy Algorithm
	\begin {enumerate}[(a)]
		\item
		\item
		\item
	\end {enumerate}
\end {enumerate}

\end{document}
